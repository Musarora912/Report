
\noindent Vim stands for Vi IMproved.  It used to be Vi IMitation, but there are so many
improvements that a name change was appropriate.  Vim is a text editor which
includes almost all the commands from the Unix program "Vi" and a lot of new
ones.  It is very useful for editing programs and other plain text. All commands 
are given with the keyboard.  This has the advantage that you can keep your 
fingers on the keyboard and your eyes on the screen.

Vim is pronounced as one word, like Jim, not vi-ai-em.  It's written with a
capital, since it's a name, again like Jim.

\subsection{Modes of vim} 
There are three modes in Vim using which we create / edit plain 
text files.
\begin{itemize}
\item Command mode: All keystrokes are interpreted as commands.
\item Insert mode: Most keystrokes are inserted as text (leaving out those with
modifier keys)
\item Visual mode: Helps to visually select some text, may be seen as a submode of
the the command mode.
\end{itemize}

\subsection{Create and save files}
These are the steps to create a document / web page using Vim:
\begin{enumerate}
\item Make a new file using command "vim <filename>" in terminal.T
\item Enter your code into the file in insert mode and save using :w in 
command mode. 
\end{enumerate}



