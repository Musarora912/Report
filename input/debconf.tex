{\bf Debconf} is a backend database, with a frontend that talks to it and presents an interface to the user. There can be many different types of frontends, from plain text to a web frontend. The frontend also talks to a special config script in the control section of a debian package, and it can talk to postinst scripts and other scripts as well, all using a special protocol. These scripts tell the frontend what values they need from the database, and the frontend asks the user questions to get those values if they aren't set. 
\begin{enumerate}
\item Template-- It contains all the notes, different questions you want to ask the user, be it boolean(true/false), multiple answer, string inputs, displaying notes, everything you want to display or ask the user. 
P.S. :make sure you leave a space in the begining of the Long Description, check the file in case you have any doubt. 
\item Config-- Next, decide what order the questions should be asked and the messages to the user should be displayed from the template file we talked about earlier.\\
Config script does all this, it has no other job than to display notes and questions from template file and take input from the user in the form of answers to the question it asked. 
\item Script-- The job of this script is to use the input stored in debconf database. It all depends how you want to use the inputs, I jave just printed the inputs stored. 
\end{enumerate}
