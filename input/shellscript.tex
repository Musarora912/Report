\subsection{Bash}
Bash is a ``Unix shell": a command-line interface for interacting with 
the operating system. It is widely available, being the default shell 
on many GNU/Linux distributions and on Mac OS X; and ports exist for 
many other systems. It was created in the late 1980s by a programmer 
named Brian Fox, working for the Free Software Foundation. It was 
intended as a free-software alternative to the Bourne shell (in fact, 
it's name is an acronym for ``Bourne-again shell"), and it incorporates 
all features of that shell, as well as new features such as integer 
arithmetic and in-process regular expressions.
\subsection{Shell}
The shell is the program which actually processes commands and returns 
output. Most shells also manage foreground and background processes, 
command history and command line editing. These features (and many more) 
are standard in bash, the most common shell in modern linux systems.
\subsection{Shell Scripting?}
In addition to the interactive mode, where the user types one command 
at a time, with immediate execution and feedback, Bash (like many other 
shells) also has the ability to run an entire script of commands, known 
as a ``Bash shell script" (or ``Bash script" or ``shell script" or just 
``script"). A script might contain just a very simple list of commands 
— or even just a single command — or it might contain functions, loops, 
conditional constructs, and all the other hallmarks of imperative 
programming. In effect, a Bash shell script is a computer program 
written in the Bash programming language.Shell scripting is the art of 
creating and maintaining such scripts.

Shell scripts can be called from the interactive command-line described 
above; or, they can be called from other parts of the system. One script 
might be set to run when the system boots up; another might be set to 
run every weekday at 2:30 AM; another might run whenever a user logs 
into the system.

Shell scripts are commonly used for many system administration tasks, 
such as performing disk backups, evaluating system logs, and so on. 
They are also commonly used as installation scripts for complex programs. 
They are particularly suited to all of these because they allow 
complexity without requiring it: if a script just needs to run two 
external programs, then it can be a two-line script, and if it needs 
all the power and decision-making ability of a Turing-complete imperative 
programming language, then it can have that as well.

\subsection{Features of Shell Scripting}
\begin{itemize}
\item Functions.
\item Arrays.
\item Commands like sed, awk.
\item Use of mysql commands through shell-importing,exporting a 
database, etc.
\end{itemize}
