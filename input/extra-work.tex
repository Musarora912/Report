There were many other tasks that I performed during my training period. 
One of task was to build another website using a template(taken from Internet) and other was writing a blog.

Here is a link to my Daily Diary Blog :
http://muskana912.wordpress.com/





\subsection{Aim Of Blogging}
Blogging is quite a trend from a few years ago. These days writing blogs has become hobby.
People write blogs for sharing their work, experience, save others time for doing the same 
work and for their Professional Resume. Companies and product owners give their
ads on different sites. There are many sites that provide facility to 
create our own blogs and easily customize them.
Blogging makes a person better thinker and writer, increases vocabulary and english grammer. It increases confidence in yourself. I wrote two blogs : One for earning and one as a daily diary. 
Wordpress blog was used a daily diary by me in training. This let me knew at the end of day what I actually do whole day.
Other Blog I have is on Blogspot, which I wish to get approved by adsense in november.
It's llnk is :
http://techystuffcse.blogspot.in/
\begin{itemize}
\item I learnt Google Docs and using Linux Operating System.
\end{itemize}

\begin{itemize}
 
\item Other thing I did was I attended many seminars which used to held at TCC. The best one was on '3-D printers' by Satyam Malhotra .
\end{itemize}
\subsection{Reporting}
I made 2-3 reports also on various seminars. One was on Entrepreneurship by Harman and Vigas.
\begin{itemize}
 
\item I learnt HTML and Python.
\end{itemize}
\subsection{What is HTML}
HyperText Markup Language (HTML) is the main markup language for creating
 web pages and other information that can be displayed in a web browser.

HTML is written in the form of HTML elements consisting of tags enclosed in
 angle brackets (like $<$html$>$), within the web page content. HTML tags most
 commonly come in pairs like $<$h1$>$ and $<$/h1$>$, although some tags represent 
empty elements and so are unpaired, for example $<$img$>$. The first tag in a 
pair is the start tag, and the second tag is the end tag (they are also 
called opening tags and closing tags). In between these tags web designers 
can add text, further tags, comments and other types of text-based content.

The purpose of a web browser is to read HTML documents and compose them into 
visible or audible web pages. The browser does not display the HTML tags, but 
uses the tags to interpret the content of the page.

HTML elements form the building blocks of all websites. HTML allows images and 
objects to be embedded and can be used to create interactive forms. It provides 
a means to create structured documents by denoting structural semantics for text 
such as headings, paragraphs, lists, links, quotes and other items. It can embed 
scripts written in languages such as JavaScript which affect the behavior of 
HTML web pages.


